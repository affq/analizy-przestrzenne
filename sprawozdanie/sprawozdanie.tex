\documentclass{article}
\usepackage[polish]{babel}
\usepackage[T1]{fontenc}
\usepackage{float}
\usepackage{subcaption}
\usepackage{caption}
\usepackage[a4paper,top=2cm,bottom=2cm,left=3cm,right=3cm,marginparwidth=1.75cm]{geometry}
\parindent = 0pt
\usepackage{amsmath}
\usepackage{graphicx}
\usepackage{hyperref}
\usepackage{listings}
\usepackage{xcolor}
\lstset{
    language=Python,                  % Ustawienie języka na Python
    basicstyle=\ttfamily\small,       % Czcionka kodu
    keywordstyle=\color{blue},        % Kolor słów kluczowych
    commentstyle=\color{green!50!black}, % Kolor komentarzy
    stringstyle=\color{red},          % Kolor stringów
    showstringspaces=false,           % Ukryj spacje w stringach
    frame=single,                     % Ramka wokół kodu
    breaklines=true,                  % Złam długie linie
    numbers=left,                     % Numery linii po lewej
    numberstyle=\tiny\color{gray}     % Styl numerów linii
}

\title{Projekt 1: Wskazanie optymalnej lokalizacji farmy fotowoltaicznej – analizy wielokryterialne (MCE)}
\author{Adrian Fabisiewicz (328935)}
\begin{document}
\maketitle
\renewcommand{\labelenumii}{\arabic{enumi}.\arabic{enumii}}
\renewcommand{\labelenumiii}{\arabic{enumi}.\arabic{enumii}.\arabic{enumiii}}
\renewcommand{\labelenumiv}{\arabic{enumi}.\arabic{enumii}.\arabic{enumiii}.\arabic{enumiv}}

\section{Wybór lokalizacji farmy fotowoltaicznej}

co należy wziąć pod uwagę, wybierając lokalizację farmy
fotowoltaicznej (rozważania teoretyczne, akty prawne wraz z cytowaniami źródeł /
bibliografią)

\section{Cel i analizowany obszar}

Celem projektu było wskazanie optymalnej lokalizacji nowej farmy fotowoltaicznej dla obszaru gminy Świeradów-Zdrój (powiat lubański, województwo dolnośląskie).

\section{Analizowane kryteria}

\begin{table}[h!]
    \centering
    \renewcommand{\arraystretch}{1.4}
    \begin{tabular}{|c|p{4cm}|p{5cm}|p{3.5cm}|}
    \hline
    \textbf{Lp} & \textbf{Kryterium} & \textbf{Parametry} & \textbf{Źródło danych do kryterium} \\ \hline
    1 & odległość od rzek i zbiorników wodnych & jak najbliżej; nieprzekraczalna 100-metrowa strefa ochronna & BDOT10k(SWRS, PTWP)\\ \hline
    2 & odległość od budynków mieszkalnych & jak najdalej, powyżej 150m & BDOT10k(BUBD)\\ \hline
    3 & pokrycie terenu & powyżej 15m od lasu, optymalnie powyżej 100m od lasu & BDOT10k(PTLZ)\\ \hline
    4 & dostęp do dróg utwardzonych & jak największe zagęszczenie & BDOT10k(SKDR)\\ \hline
    5 & nachylenie stoków & jak najbardziej płasko & NMT\\ \hline
    6 & dostęp światła słonecznego & optymalnie: stoki południowe (SW-SE) & NMT\\ \hline
    7 & dobry dojazd od istotnych drogowych węzłów komunikacyjnych & jak najkrótszy czas dojazdu & BDOT10k(SKDR)\\ \hline
    \multicolumn{4}{|c|}{Łączenie kryteriów} \\ \hline
    8 & ocena przydatności terenu (próg przydatności) & \multicolumn{2}{|c|}{80\% / 90\% max. przydatności}\\ \hline
    9 & przydatne działki / grupy działek & min 60\% działki na terenie przydatnym & EGIB \\ \hline
    10 & powierzchnia i min. szerokość obszaru & \multicolumn{2}{|c|}{2ha / 50m}\\ \hline
    11 & koszt przyłącza do sieci SN (mapy kosztów) & jak najniższy & BDOT10k (wszystkie warstwy PT)\\ \hline
    \end{tabular}
    \caption{Tabela z kryteriami lokalizacji}
    \label{tab:kryteria}
    \end{table}


\section{Realizacja}
\subsection{Ustalenie środowiska pracy i ścieżek do danych}
\begin{lstlisting}
import arcpy.analysis
import arcpy.management
import arcpy.sa

geobaza = r"C:\Users\adria\Desktop\STUDIA_FOLDERY\analizy\MyProject12\MyProject12.gdb"
arcpy.env.workspace = "in_memory"
arcpy.env.outputCoordinateSystem = arcpy.SpatialReference("ETRS_1989_Poland_CS92")
arcpy.env.extent = f"{geobaza}\\gmina_buffer"
arcpy.env.mask = f"{geobaza}\\gmina_buffer"
arcpy.env.cellSize = 5
arcpy.env.overwriteOutput = True

swrs_0210_buffer = arcpy.analysis.Buffer(f'{geobaza}\\SWRS_L_0210', f'{geobaza}\\SWRS_L_0210_buffer', '1 Centimeter')
swrs_0212_buffer = arcpy.analysis.Buffer(f'{geobaza}\\SWRS_L_0212', f'{geobaza}\\SWRS_L_0212_buffer', '1 Centimeter')
water = arcpy.management.Merge([swrs_0210_buffer, swrs_0212_buffer, f'{geobaza}\\PTWP_A_0210', f'{geobaza}\\PTWP_A_0212'], 'water')
budynki = arcpy.management.Merge([f'{geobaza}\\BUBD_A_0210', f'{geobaza}\\BUBD_A_0212'], 'budynki')
ptlz = arcpy.management.Merge([f'{geobaza}\\PTLZ_A_0210', f'{geobaza}\\PTLZ_A_0212'], 'ptlz')
nmt = f'{geobaza}\\nmt'
drogi = arcpy.management.Merge([f'{geobaza}\\SKDR_L_0210', f'{geobaza}\\SKDR_L_0212'], 'drogi')
wezly = f'{geobaza}\\wezly_raster'
dzialki = f'{geobaza}\\dzialki'
pt_merged = f'{geobaza}\\PT_merged'
linie_elektroenergetyczne = arcpy.management.Merge([f'{geobaza}\\SULN_L_0210', f'{geobaza}\\SULN_L_0212'], 'linie_elektroenergetyczne')
\end{lstlisting}

\subsection{Kryterium 1}
\subsubsection{Kod}
\begin{lstlisting}
\end{lstlisting}

\subsubsection{Wynik}

\subsection{Kryterium 2}
\subsubsection{Kod}
\begin{lstlisting}
\end{lstlisting}

\subsubsection{Wynik}

\subsection{Kryterium 3}
\subsubsection{Kod}
\begin{lstlisting}
\end{lstlisting}

\subsubsection{Wynik}

\subsection{Kryterium 4}
\subsubsection{Kod}
\begin{lstlisting}
\end{lstlisting}

\subsubsection{Wynik}

\subsection{Kryterium 5}
\subsubsection{Kod}
\begin{lstlisting}
\end{lstlisting}

\subsubsection{Wynik}

\subsection{Kryterium 6}
\subsubsection{Kod}
\begin{lstlisting}
\end{lstlisting}

\subsubsection{Wynik}

\subsection{Kryterium 7}
\subsubsection{Kod}
\begin{lstlisting}
\end{lstlisting}

\subsubsection{Wynik}

% \begin{figure}[H]
%     \centering
%     \includegraphics[width=0.75\textwidth]{img/interfejs.png}
%     \caption*{}
% \end{figure}

\end{document}